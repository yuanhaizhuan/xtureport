%% 美赛模板:正文部分

\documentclass[12pt]{article}  % 官方要求字号不小于 12 号,此处选择 12 号字体
\usepackage{xtureport}  % 载入 Xtureport 模板文件

\title{研究生学术科研工作周汇报模板}  % 标题
\author{袁海专}
\weekno{1}  %%% 第1次汇报


\begin{document}

\maketitle 
%\tableofcontents  % 生成目录


% 正文开始
\section{中文示例}
\subsection{代码示例}
\lstinputlisting[language=C++]{./codes/xtureport-matlab.m}

\subsection{公式示例}
\begin{equation}\label{eq:heat}
\frac{\partial u}{\partial t} - a^2 \left( \frac{\partial^2 u}{\partial x^2} + \frac{\partial^2 u}{\partial y^2} + \frac{\partial^2 u}{\partial z^2} \right) = f(x, y, z, t)
\end{equation}

\subsection{插图示例}
\begin{figure}[htbp]
  \centering
  \includegraphics[width=.2\textwidth]{./figures/xtu-fig-logo.pdf} \quad 
  \includegraphics[width=.6\textwidth]{./figures/xtu-text-logo.pdf}
\caption{湘潭大学}\label{fig:result}
\end{figure}

\subsection{列表示例}
\begin{enumerate}[\bfseries 1.]
    \item 我们做了 ...
    \item 我们做了 ...
    \item 我们做了 ...
\end{enumerate}

\subsection{表格示例}
\begin{table}[!htbp]
\begin{center}
\caption{Notations}
\begin{tabular}{cl}
	\toprule
	\multicolumn{1}{m{3cm}}{\centering Symbol}
	&\multicolumn{1}{m{8cm}}{\centering Definition}\\
	\midrule
	$A$&the first one\\
	$b$&the second one\\
	$\alpha$ &the last one\\
	\bottomrule
\end{tabular}\label{tb:notation}
\end{center}
\end{table}


\section{Demo in English}
\subsection{Mathematical Model}
The detail can be described by equation \eqref{eq:heat}:
\begin{equation}\label{eq:heat}
\frac{\partial u}{\partial t} - a^2 \left( \frac{\partial^2 u}{\partial x^2} + \frac{\partial^2 u}{\partial y^2} + \frac{\partial^2 u}{\partial z^2} \right) = f(x, y, z, t)
\end{equation}

\subsection{Conlusions}
The results are shown in Figure \ref{fig:result}, ... 


% 以下为信件/备忘录部分,不需要可自行去掉
% 如有需要可将整个 letter 环境移动到文章开头或中间
% 请在后一个花括号内填写信件(Letter)或备忘录(Memorandum)标题
%\begin{letter}{Memorandum}
%\begin{flushleft}  % 左对齐环境,无首行缩进
%\textbf{To:} Heishan Yan\\
%\textbf{From:} Team XXXXXXX\\
%\textbf{Date:} October 1st, 2019\\
%\textbf{Subject:} A better choice than MS Word: \LaTeX
%\end{flushleft}
%In the memo, we 
%Firstly, \ldots
%\end{letter}


% 参考文献,此处以 MLA 引用格式为例
\begin{thebibliography}{99}
\bibitem{1} Einstein, A., Podolsky, B., \& Rosen, N. (1935). Can quantum-mechanical description of physical reality be considered complete?. \emph{Physical review}, 47(10), 777.
\bibitem{2} \emph{A simple, easy \LaTeX\ template for MCM/ICM: Xtureport}. (2018). Retrieved December 1, 2019, from\url{https://www.cnblogs.com/xjtu-blacksmith/p/xtureport.html}
\end{thebibliography}


% 以下为附录内容
% 如您的论文中不需要附录,请自行删除
%\begin{subappendices}  % 附录环境
%\section{Appendix A: Further on \LaTeX}
%\end{subappendices}

\end{document}  % 结束
